%#Split: 05_abilities_daihyo % extract 
%#PieceName: p05_abilities_daihyo
% p05_abilities_daihyo_00.tex
\KLBeginSubject{04}{4}{研究者調書(研究代表者)}{2}{F}{1}{kiban-s-pub-subject-header}{kiban-s-pub-header}

\section{研究者調書(研究代表者)}
%    <<最大 2ページ>>

% s14_abilities_S
% s14_pub_S_commands
% begin 研究者情報 ==========================
\renewcommand{\研究者氏名}{中村浩二}
\renewcommand{\研究者氏名ふりがな}{なかむら こうじ}
\renewcommand{\研究者生年月日の年}{1981}
\renewcommand{\研究者生年月日の月}{8}
\renewcommand{\研究者生年月日の日}{4}
\renewcommand{\研究者年齢}{41}
\renewcommand{\研究者所属機関部局職}{\small{高エネルギー加速器研究機構・素粒子原子核研究所・助教}}	% use \tiny if necessary
\renewcommand{\研究者学位}{博士(理学)}
% end 研究者情報 ==========================

% s14_pub_s_picture
\renewcommand{\KLXi}{117}
\renewcommand{\KLYi}{-27}
\renewcommand{\KLYii}{-39}

\thiswatermark{%
	\begin{picture}(0,0)(0,0)
		\put(\KLXi, -24){\tiny{(\研究者氏名ふりがな})}
		\put(\KLXi, \KLYii){\研究者氏名}
		\put(375, \KLYi){\small{\研究者生年月日の年}}
		\put(410, \KLYi){\small{\研究者生年月日の月}}
		\put(435, \KLYi){\small{\研究者生年月日の日}}
		\put(410, \KLYii){\small{\研究者年齢}}
		\put(116, -52){%
			\parbox[t][50pt]{143pt}{\研究者所属機関部局職}
		}
		\put(330, -58){\研究者学位}
	\end{picture}
}
 % pieces

%\PapersInstructions	% <-- 留意事項。これは消すか、コメントアウトしてください。
%begin 基盤S研究代表者の研究遂行能力及び研究環境 ====================
	\noindent
	\textbf{(1)研究代表者のこれまでの研究活動}\\
2009年に米国フェルミ国立加速器研究所で行われていたCDF実験における単一トップクォーク生成過程の発見および小林益川行列の第三世代要素の測定を主題とする博士(理学)を取得した後、博士研究員(東京大学)からATLAS 実験に参加し、ヒッグス物理解析を行い、ヒッグス粒子の発見に貢献した~\cite{higgsobs}。発見後にはフェルミ粒子に崩壊するヒッグス粒子の発見を解析グループのコーディネータとして主導した~\cite{higgstau,higgshunting2015}。2012 年に高エネルギー加速器研究機構への異動後はHL-LHC 用内部飛跡検出器の開発研究を進めてきた。浜松ホトニクス社と共同でプラナ型のピクセルモジュールの開発および量産を主導・推進している~\cite{atlaspixel1,atlaspixel2}。現在はHL-LHC 用ピクセル検出器のセンサーグループのコーディネータとして量産に向けた市場調査、最終デザインの決定に向けた準備をリードしている。また、高時間分解能検出器(LGAD)に関して初期から浜松ホトニクス社と共同で開発を進めると共にその可能性をいち早く認識し、将来のハドロン衝突型加速器への応用の可能性を探ってきた~\cite{lgadvertex2022,lgadnim2023,vertex2020}。
\MEMO{研究計画に関連した国際的な取組(国際共同研究の実施歴や海外機関
での研究歴等)がある場合には必要に応じてその内容を含めること。}
\\

	\noindent
	\textbf{(2)研究代表者の研究環境}\\
研究代表者はHL-LHC 用ピクセル検出器の開発および量産準備を主導・推進しているため、半導体検出器開発に必要なすべての設備は高エネルギー加速器研究機構に整備されている。具体的には高エネルギー加速器研究機構富士実験棟に約40m2 のクリーンルームが二つあり、そのうち一つは研究代表者が設計し管理しているものである。次に示すような必要な機材はすべてクリーンルームに配備されている。全自動ワイヤーボンダー、セミオートプローブステーション、双眼実体顕微鏡、Nd:YAGレーザー、デシケータ、恒温槽、高圧電源、低圧電源、LCRメータ、測定顕微鏡、3次元測定器、はんだステーション、簡易基板製造機など。また、検出器の性能評価に必要な照射試験やテストビームラインの使用経験が豊富で、東北大およびフェルミ研究所での必要な実験に支障は無い。さらに私が高エネルギー加速器研究機構に着任してから現在までのすべての実験の方法に関して私が管理するWeb上に研究資料として存在する。(\url{http://atlaspc5.kek.jp/do/view/Main/WebHome})
\\

	\noindent
	\textbf{(3)研究組織全体の研究環境}\\

増渕と中村はヒッグス粒子発見以前からヒッグス粒子に

中村と廣瀬が所属する筑波大学の学生はLGAD検出器の開発を開発当初から共同で行っており、共同研究の面でも研究設備の面でも開発の体制が確立されている。

以上の観点から、研究組織全体の研究環境は高輝度LHC実験後期検出器の設計書を作成するうえで全く問題がない。

%end 基盤S研究代表者の研究遂行能力及び研究環境 ====================

%\EnvironmentInstructions	% <-- 留意事項。これは消すか、コメントアウトしてください。
%begin 研究業績リスト ====================
\renewcommand{\refname}{研究業績リスト}
\begin{thebibliography}{99}
    \bibitem{lgadvertex2022}(査読中) {\bf Development of AC-LGAD detector with finer pitch electrodes for high energy physics experiment}, S.~Kita, \me ~\etal  JPS Conference Proceeding, Proceedings of the 31st International Workshop on Vertex Detectors (VERTEX), submitted.
    \bibitem{lgadnim2023}(査読有り) {\bf Optimization of capacitive coupled Low Gain Avalanche Diode (AC-LGAD) sensors for precise time and spatial resolution}, S.~Kita, \me ~\etal  Nucl. Instrum. Methods Phys. Res., Sect. A,  Volume 1048, March 2023, 168009
    \bibitem{vertex2020}(招待講演) {\bf First prototype of finely segmented HPK AC-LGAD detectors}, The 29th International Workshop on Vertex Detectors, 5-8 October 2020, Online conference.
%          \bibitem{lgadnim2018}(査読有り) {\bf Evaluation of characteristics of Hamamatsu low-gain avalanche detectors}, S.~Wada, K.~Hara, \me ~\etal  Nucl. Instrum. Methods Phys. Res., Sect. A, DOI:10.1016/j.nima.2018.09.143 Available online 5 October 2018.
          \bibitem{atlaspixel1} (査読有り) {\bf Development of the radiation tolerant fine pitch planar pixel detector by HPK/KEK}, \me ~\etal , Nucl. Instrum. Methods Phys. Res., Sect. A, DOI:10.1016/j.nima.2018.09.015 Available online 8 September 2018.
          \bibitem{atlaspixel2} (査読あり) {\bf Irradiation and Testbeam of KEK/HPK Planar p-type Pixel Modules for HL-LHC}, \me ~\etal, proceeding of International Workshop on Semiconductor Pixel Detectors for Particles and Imaging(PIXEL2014), Journal of Instrumentation, Volume {\bf 10}, June 2015 
          \bibitem{higgstau} (査読あり) {\bf Evidence for the Higgs-boson Yukawa coupling to tau leptons with the ATLAS detector}, \me ~\etal [ATLAS Collaboration], Journal of High Energy Physics {\bf 117} (2015)
         \bibitem{higgshunting2015} (講演) {\bf Fermion Coupling to Higgs in ATLAS}, \me, Higgs Hunting 2015, LAL, Olsay
%        \item (査読あり) {\bf ``Search for Standard Model Higgs boson in $H\to\tau\tau$ decay with the ATLAS detector''},\me  ~\etal (Proceeding of the Hadron Collider Physics Symposium (HCP 2012)) EPJ Web of Conferences {\bf 49}, 2013
        
%	\hline%----------------------------------------------
%	2012 \\
        \bibitem{HCP2012} (講演) {\bf Search for Standard Model Higgs boson in $H\to\tau\tau$ decay with the ATLAS detector}, \me, Hadron Collider Physics Symposium (HCP 2012), Kyoto University, Japan 12-16 November, 2012.  
          
%        \item (招待講演) {\bf Latest results on the Standard Model Higgs bosons in ATLAS }, \me, KEK物理セミナー, 2012年11月20日, 高エネルギー加速器研究機構 

%        \item (招待講演) {\bf Latest Results on the Standard Model Higgs Searches at the LHC }, \me, IPMUセミナー, 2012年7月13日, カブリ数物連携宇宙研究機構

%        \item (招待講演) {\bf Latest Results on the Standard Model Higgs Searches at the LHC }, \me, KEK物理セミナー, 2012年7月12日, 高エネルギー加速器研究機構

        \bibitem{higgsobs} (査読あり) {\bf Observation of a new particle in the search for the Standard Model Higgs boson with the ATLAS detector at the LHC},
	  \me ~\etal [ATLAS Collaboration], 
	  Phys. Lett. B {\bf 716}, 1-29 (2012)

\end{thebibliography}

%end 研究業績リスト ====================

% p05_abilities_daihyo_01.tex
\KLEndSubject{F}


